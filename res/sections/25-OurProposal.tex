\section{Nduja}
%~ Here we could explain our idea. We could compare it with previous works.
%~ Add evaluation if we implement something

\subsection{Data retrieving}
Data that we used to implement our idea is publicly available on the Web. In
fact to bind physical people to wallet addresses we exploit the fact that
people frequently advertise their own public keys to receive founds
through donations.

We build our data set by searching onto social networks and on developers'
repositories (e.g. github) and then expanding it with different techniques
explained in other studies~\cite{fistful}. Following our idea, after discoving
a wallet we could certainly associate it with an account that could have some
other public information. Because a person could have more that a single wallet
we search for all transaction in which the wallet is involved and we suppose
that all the addresses input of a certain transaction are under control of the
same person.

\subsubsection*{Twitter}
We have chosen this social network because most \textit{tweets} of the users are
public and because there are several people that ask for donations 
for disparate purposes or the owners of some accounts decide to donate some
Bitcoins or altcoins to gain followers and popularity (e.g., using hashtag
\texttt{\#BTCGiveAway}). They advertise their intent with a tweet, in which
they ask the readers to follow them and reply with their own altcoin wallet.
After some time the owner draw from the replies one of the authors and performs
the donation.

\subsubsection*{Repositories}
Free-time developers that distribute their own software with free licenses may
ask the users to contribute to their project with some Bitcoins or altcoins to
guarantee the further development of their tools. The choice of targeting our
research even on developers is because they are more likely to follow new
technologies and proposal such as new criptocurrencies.

Both these resources provide well-documented search APIs~\cite{}~\cite{}.

\subsection{Methodology}
\label{sec:methodology}
We define a set of altcoins that we would like to de-anonymize. For each of them we define the symbol, the name and a regular expression
that matches the wallet addresses for a particular coin.

Based on this information we build queries as explained in section~\ref{sec:queries}.

\subsection{Queries} 
\label{sec:queries}

To maximize the number of results


