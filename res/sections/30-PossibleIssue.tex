\section{Issues} \label{sec:issues}
During the collecting phase we inserted into the database a large number of
false positives, especially Ethereum addresses. This is due to the fact that
Ethereum addresses are simply 20 bytes hexadecimal number that could appear
also without the \texttt{0x} prefix. We filter out the majority of false
positives with \texttt{address\_checker} module. We spent a good deal of time
cleaning the database manually. Finally, we kept only about the 45\% of the
original addresses. 

Another issue is the fact that we had at our dispose a list of known services'
addresses only for Bitcoin and not for the others.

We clustered only Litecoin and Dogecoin addresses for the following reasons.
First of all the clustering using web APIs is extremely slow, because their
rate limits: in fact Dogecoin clustering took about one week, Litecoin 5 days.
We desisted from doing the same for Bitcoin because the starting number of
wallets for this currency was even higher and its blockchain is even bigger. We
also tried to perform this search locally downloading Dogecoin and Litecoin
clockchains, but their official clients (derived from \texttt{bitcoind}) do not
construct the same mappings as the web services we used. We tried two
approaches: on the one side we tried to build the mapping wallet-transactions
and on the other side we exploited the possibility to import so called
``watch-only'' addresses making us able to access directly all the transaction
of all the addresses imported. The first approach was abandoned because it
requires too much time and the data were too huge. The latter because of the
continuous rescanning of the blockchain each time new wallets were imported.
That obliged us to restart the client multiple times during a single run and
this brought to a client misbehavior. Also the rescanning phase is a really
time consuming task.

