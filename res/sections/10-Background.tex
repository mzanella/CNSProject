\section{Background}
\subsection{Blockchain and Bitcoin}
Bitcoin was the first cryptocurrency developed that makes use of the
Blockchain technology. Bitcoin and Blockchain rely on the digital
signatures to prevent the repudiation of a transaction, without a third
party check.
To achieve this, it was designed a peer-to-peer payment system,
in which each peer maintain a copy of the complete transaction history.
A transaction is a transfer of value from one wallet to another one.
Once a transaction is built and signed by the sender, it is broadcasted.
Some nodes of the bitcoin network, called \textit{miners}, collect the
transactions into blocks and try to find a difficult
proof-of-work~\cite{pricing}~\cite{hashcash} for the
block, i.e., they need to find a nonce such that the hash of the block
contains a certain number of leading zeros.
If one miner succeeds it broadcast the block to the network.
The remaining nodes check if the block's transactions are valid and
whether they were already spent or not.
Since this operation is computationally difficult the successful
generation of a block is rewarded.

The blockchain technology becomes popular because it allows
to prevent double spending problem without a third party by using
transparent transactions.
The bitcoin system preserves user personal privacy, giving them some
sort of pseudo-anonymity.
In this system every user owns at least one wallet, that is associated
with an address. Addresses are 26-35 alphanumeric characters long and
are indispensable in order to be able to exchange Bitcoins.
In fact to transfer Bitcoin to another person you have to know her
wallet address, that works as user account.
It is also possible to send money to non-existent addresses: when
someone will create a new wallet with that address she could redeem
them.
Each user could generate more wallets and they do not require any kind
of authentication\todo{Cosa intendi?}.
It is recommended to generate a new address for each transaction
to better preserve the anonymity~\cite{satoshi}.
In order to use a certain address the owner of the wallet must know
the private key associate with it, in fact if it
will be lost there is no possibility to recover it.


\subsection{Altcoins}
After the success of Bitcoin other digital currencies took place.
All these cryptocurrencies that are different from Bitcoin are called
\emph{altcoins} (Alternative Coins)~\cite{bitcoinbeyond}.
One of the most famous is
\textbf{Ethereum}. The peculiar characteristic of this cryptocurrency is
the language built-in in the blockchain. It allows the developers to
write their own smart-contracts and decentralized
applications~\cite{bib:ethereum:whitepaper}\todo{Explain the concept of
smart-contracts and decentralized application?}.
\todo{Do we keep the Ethereum Classic Stuff?}
Lots of aspects are similar to Bitcoin: even
this altcoin is based on blockchain, transactions are public, users are
pseudo-anonymous. In Ethereum wallets are associated with 40
alphanumeric characters addresses that could has as prefix \texttt{0x}.
Two other altcoins that have public transactions and were considered
throughout our work are: \textbf{Litecoin} and \textbf{Dogecoin}.
The first one is a cryptocurrency born in 2011 with the aim to improve
Bitcoin and thought to be ``lighter'' to create blocks.
They changed the proof of work algorithm to permits the
generation of a new block every 2.5 minutes rather than 10 minutes as in
bitcoin~\cite{bib:litecoin:wiki}.
Litecoin wallets starts with \texttt{L} or \texttt{M} followed by an
alphanumeric string whose length is between 24 and 36 characters.
The latter is an altcoin derived from Litecoin, born as a joke in 2013
but that rapidly increased its market capitalization, exceeding 1
billion dollars in January 2018~\cite{}. Dogecoin addresses
starts with \texttt{D} followed by 33 alphanumeric characters.
A slightly different altcoin is \textbf{Monero}: this currency was
created with the aim of privacy, so the transaction chain is not public
and there is no way to know how much money a wallet has collected and
with which other wallets it sent transactions.
Monero wallets are a 90 alphanumeric addresses that starts with
\texttt{4}.


\subsection{Pseudo-anonymity in Bitcoin system}
We can define a subject as anonymous if the subject is not identifiable
within a set of subjects, denoted as the anonymity
set~\cite{terminology}. Bitcoin system does not provide it
because~\cite{deanon}
\begin{enumerate*}[label=\roman*),itemjoin={,\quad}]
\item The real-name authentication mechanism helps Bitcoin service
providers to find the addresses that ever deposited and withdrew
\item a Bitcoin wallet advertised on Internet can be related to its
owner
\item the chain of transactions is transparent
\item exchanging inputs from an address to another one could expose the
sender
\item the change address of transactions could be classified by
attackers to the sender.
\end{enumerate*}
So in that context we can speak about pseudo-anonymity: until a user
does not associate personal information with a certain wallet, she
could be considered anonymous.
