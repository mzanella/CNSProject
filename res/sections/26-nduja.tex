\newcommand{\walletcollector}{\texttt{wallet\_collector}}
\newcommand{\userinforetriever}{\texttt{user\_info\_retriever}}
\newcommand{\addresschecker}{\texttt{address\_checker}}
\newcommand{\graph}{\texttt{graph}}

\section{Nduja}
\subsection{High level architecture}
Nduja was developed in Python\footnote{it should be used with Python3.5+} and it
is composed by four main modules, represented in Figure\ref{fig:architecture}:
\begin{enumerate*}[label=\roman*),itemjoin={,\quad}]
\item \walletcollector{} is a set of classes whose aim is to retrieve
addresses of different cryptocurrencies. It provides a different class for each
different sources that we used, following the same interface
\item \userinforetriever{} is a set of classes that, based on which
site a wallet is discovered, retrieve some information (e.g. name, email,
personal website) related to the owner of that wallet
\item \addresschecker{} is a module that have to check if a sequence of
characters recognized as an address is a real and used addresses. For certain
altcoins it is not possible to perform this validation action accurately: Monero
has private transactions and for Ethereum there is not an efficient API to find
out that information
\item \graph{} is the module that create graph with which is possible to
correlate addresses and plot graphs.
\end{enumerate*}

\begin{figure}[H]
\centering
\includegraphics[scale=0.2]{db}
\caption{Nduja high level architecture}
\label{fig:architecture}
\end{figure}

\subsection{Queries} 
\label{sec:queries}
In order to maximize the number of possible results in repositories the module
\walletcollector{} searches for files that contains words such as
\textbf{donation} or \textbf{contribution} and in which there is a set of
characters that matches regular expressions that we defined for different
address formats. This has been implemented using two different APIs: Github
Search API \footnote{\url{https://developer.github.com/v3/search/}} and
Searchcode API \footnote{\url{https://searchcode.com/api/}}.
Instead, in order to query Twitter we looked for the same keywords that in the
repositories but also for \textit{hashtags} such as \textbf{\#GiveAway} or for
hashtags correlated to specific cryptocurrency (e.g. \#BTCGiveAway). This has
been done using Twitter API
\footnote{\url{https://developer.twitter.com/en/docs/api-reference-index}}.
Twitter API has a strong limitation: we are only able to retrieve tweets that
are published at most a week before the search. If there was a possibility to
bypass this limitation it could be possible to retrieve even more information.

\subsection{Database}
All data we have retrieved is stored into a
SQLite\footnote{\url{https://www.sqlite.org/}} database. Addresses are save with
the indication that if they are wallets retrieved from the websites we search,
if they are inferred or they are ``tagged'' addresses. These are wallets that
are known a priori: that means that are used by famous services, such as mixing
services or betting games as
\textit{SatoshiDICE}\footnote{\url{satoshidice.com}}. We excluded that kind of
services from data retrieving process for two reasons: the first is that we did
not want that the number of addresses explode, the latter is that these
addresses are ``public'' yet, so they are useless for the aim of our study.
