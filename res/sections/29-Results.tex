% SELECT COUNT(*) FROM Wallet WHERE inferred=0; (before clustering)
\newcommand{\startingNumberAllWallets}{10477}

% SELECT COUNT(*) FROM Wallet WHERE Status != -1 and inferred=0; (before clustering)
\newcommand{\startingNumberWalletsNotService}{6603}

% SELECT COUNT(*) FROM Wallet WHERE Status = 1; (before clustering)
\newcommand{\startingNumberWalletsAtLeastOneTransaction}{1741}

% SELECT COUNT(*) FROM Wallet WHERE 1; (after clustering)
\newcommand{\clusteringNumberAllWallets}{16910}

% SELECT COUNT(*) FROM Wallet WHERE Status != -1; (after clustering)
\newcommand{\clusteringNumberWalletsNotService}{13036}

% SELECT COUNT(*) FROM Wallet WHERE Status != -1 AND Currency='BTC' and inferred=0;
% (before clustering)
\newcommand{\startingBTC}{1953}

% SELECT COUNT(*) FROM Wallet WHERE Status != -1 AND Currency='LTC' and inferred=0;
% (before clustering)
\newcommand{\startingLTC}{813}

% SELECT COUNT(*) FROM Wallet WHERE Status != -1 AND Currency='XMR' and inferred=0 ;
\newcommand{\startingXMR}{228}

% SELECT COUNT(*) FROM Wallet WHERE Status != -1 AND Currency='ETH' and inferred=0;
\newcommand{\startingETH}{2953}

% SELECT COUNT(*) FROM Account WHERE 1;
\newcommand{\accountNumber}{3444}

% SELECT (x*1.0)/(y*1.0) FROM
% (
%     SELECT COUNT(*) AS x
%     FROM Wallet
%     WHERE Wallet.Status != -1
% ),
% (
%     SELECT COUNT(*) AS y
%     FROM Account
% )
\newcommand{\avarageAccount}{$~$3.785}

%~ SELECT COUNT(*) FROM (SELECT DISTINCT * FROM AccountRelated)
\newcommand{\accountRelated}{275}


\section{Results} \label{results}
After we implemented and tested \texttt{Nduja} we run it. The first step of the
algorithm give us \startingNumberAllWallets{} addresses,
\startingNumberWalletsNotService{} addresses that probably are not services,
but only \startingNumberWalletsAtLeastOneTransaction{} of them we were able to
attest that they had at least one trasaction. Finally, after the clustering,
we increase this number to \clusteringNumberAllWallets{}, with
\clusteringNumberWalletsNotService{} wallets that could be associate with
services. In Figure~?!? is depicted the increase of wallet we found out.
In Figure~?!? we can see how the addresses that we collected are distributed
in the different currencies. As we expected the number of Monero and Ethereum
wallets are lower than the other: Monero has private transactions, so it is not
possible to analyze its blockchain, instead Ethereum does not allow
multiple-input transactions, so it is not possible to cluster wallets as for
the others. The number of Monero wallets found is \startingXMR{} and it could
be so low even because it is an altcoin not so famous and it was born in order
to preserve users privacy, so people could be more aware to advertise it.
Addresses are connected to \accountNumber{} accounts, so most of them own more
than one wallet, on average \avarageAccount{} addresses per person.
As we expected the currency on which
we could retrieve more data is Bitcoin yet. Analyzing data collected we discover
that for ?!? accounts we are able to connect a Twitter, Facebook or Linkedin
account. The least social network is the most interesting, because people share
more sensitive data on that because of its aim. In Figure~?!? this data are
represented. Finally, ?!? accounts in our database are strongly related with at
least another one: this means that we have discovered more than an account for
a certain person or that are people that are in some way related.
