\section{Issues} \label{sec:issues}
During the collecting phase we inserted into the database a large number of
false positives, especially Ethereum addresses. This is due, in particular, to
the fact that Ethereum addresses are simply 20 bytes hexadecimal number that
could appear also without the \texttt{0x} prefix. Another problem is that smart
contract addresses and Ethereum wallet public keys have the same format. To
filter out the as much false result as possible we start the search phase
inserting all well known addresses into the database flagging them. In that way
we avoid to erroneously mark someone as owner of that wallet, even if she is
correlated with it. The majority of false positive has been skimmed by
\texttt{address\_checker} module. Moreover, we eliminated all addresses coming
from \texttt{.ipynb} and image files: they were all false matches that
respected Ethereum addresses format. Lastly, we spent a good deal of time
cleaning the database manually: we focused in searching account that had a too
high number of wallet correlated or files that contained a really large set of
addresses. Finally, before starting the clustering phase, we kept only about
the 45\% of the original addresses.
Another issue is the fact that we had at our dispose addresses
of well-known services only for Bitcoin and not for the other altcoins.
We clustered only Litecoin and Dogecoin addresses for the following reasons.
First of all the clustering phase using Web APIs is extremely slow, because of
their rate limits: in fact Dogecoin clustering took about one week, Litecoin
five days.
We desisted from doing the same for Bitcoin because the starting number of
wallets for this currency was even higher and its blockchain is even bigger. We
also tried to perform this search locally downloading Dogecoin and Litecoin
blockchains, but their official clients (derived from \texttt{bitcoind}) do not
construct the same mappings as the Web services we used. We tried two
approaches: on the one side we tried to build the mapping wallet-transactions
and on the other side we exploited the possibility to import so called
``watch-only'' addresses making us able to access directly all the transaction
of addresses imported. The first approach was abandoned because it
required too much time and the data were too huge. The latter because of the
continuous rescanning of the blockchain each time new wallets were imported.
That obliged us to restart the client multiple times during a single run and
this led to a client misbehavior. Also the rescanning phase is a really
time consuming task.
