\section{Introduction}
Blockchain was introduced the first time in 2008, by Satoshi Nakamoto, an
anonymous programmer (currently, it is probably a group of people) that
designed it with the aim to create a virtual and decentralized currency,
Bitcoin~\cite{satoshi}. Since then, this technology has been developed and
studied in different fields than the financial one, for managing documents, in
the notary public, as decentralized storage and so on~\cite{air}. Bitcoin grow
rapidly and lots of other cryptocurrencies came out: lots of them are really
similar to Bitcoin, and try to improve some aspects of that system. Some
altcoins try to make the system ''lighter'', some other try to improve the
privacy, some other are only thought as an alternative. The success of these
currecy could be traced back to the possibility to generate it without any work,
their anonymity (or pseudo-anonymity) and because people are every day becoming
more familiar with this new technologies.

Alongside blockchain and its application, different kind of attacks where
developed. These malicious behaviour pricipally focus on the deanonymization of
Bitcoin (or other currencies) wallets: in layman's terms it means to find out
who owns a certain wallet. A typology of attacks concentrate on the fact that
Bitcoin and Blockchain rely on Internet infrastruction and try to deduce the
identity of a certain wallet address user by the IP address~\cite{deanon}.
Another kind of approach is to concentrate on transation graph and infer data,
for example to find out if wallets are owned to the same person~\cite{fistful}.

In this paper we exploit the fact that some people does not follow the common
rules in order to preserve the anonymity of their wallets associating to them
their pesonal data, whitout knowing that they risk to compromize even other
wallets that they own.


\subsection{Paper structure}
This paper is organized as follow: Section II...